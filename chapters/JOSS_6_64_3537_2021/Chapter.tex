\openleft%
\chapter{lattice-symmetries: A package for working with quantum many-body bases}\label{ch:ls21}
% authors:
%   - name: Tom Westerhout
%     orcid: 0000-0003-0200-2686
%     affiliation: 1
% affiliations:
%   - name: Institute for Molecules and Materials, Radboud University
%     index: 1
% date: 15 March 2021

\AdaptedFrom{https://doi.org/10.21105/joss.03537}{T. Westerhout,\\Journal of Open Source Software, 6(64), 3537 (2021).}
\clearpage

\section{Summary}

\lettrine[lines=3]{E}{xact} diagonalization (ED) is one of the most reliable and established numerical methods of quantum many-body theory. It is precise, unbiased, and general enough to be applicable to a huge variety of problems in condensed matter physics. Mathematically, ED is a linear algebra problem involving a matrix called the Hamiltonian. For a system of spin-1/2 particles, the size of this matrix scales exponentially (as $\mathcal{O}(2^N)$) with the number of particles, $N$.

Very fast scaling of memory requirements with system size is the main computational challenge of the method. There are a few techniques allowing one to lower the amount of storage used by the Hamiltonian. For example, one can store only the nonzero elements of the Hamiltonian. This is beneficial when the Hamiltonian is sparse, which is usually the case in condensed matter physics. One can even take it one step further and avoid storing the matrix altogether by instead computing matrix elements on the fly.

A complementary approach to reduce memory requirements is to make use of system symmetries. For example, many relevant Hamiltonians possess $U(1)$ symmetry, which permits one to perform calculations assuming that the number of particles (or number of spins pointing upwards) is fixed. Another example would be translational invariance of the underlying lattice.

Although the algorithms for dealing with lattice symmetries are well known~\cite{Sandvi2010ComputationalS}, implementing them remains a non-trivial task. Here we present lattice-symmetries, a package providing high-quality and high-performance realization of these algorithms. Instead of writing their own optimized implementation for every system of interest, a domain expert provides system-specific details (such as the number of particles or momentum quantum number) to lattice-symmetries and it will automatically construct a reduced Hamiltonian. The dimension of the new Hamiltonian can be multiple orders of magnitude smaller than of the original one.

Furthermore, in lattice-symmetries the Hamiltonian itself is never stored. Instead, its matrix elements are computed on the fly, which reduces the memory requirements even more. Care is taken to keep the implementation generic such that different physical systems can be studied, but without sacrificing performance as we will show in the next section.

All in all, lattice-symmetries serves as a foundation for building state-of-the-art ED and VMC (Variational Monte Carlo) applications. For example, SpinED~\cite{SpinED} is an easy-to-use application for exact diagonalization that is built on top of `lattice-symmetries` and can handle clusters of at least 42 spins on a single node.

\section{Statement of need}

Exact diagonalization is an old and well-established method and many packages have been written for it. However, we find that most state-of-the-art implementations~\cite{Wietek2018subla,Lauchl2019Kagome} are closed-source. There are only three notable open-source projects that natively support spin systems\footnote{%
There are a few projects targeting fermionic systems and the Hubbard model in particular. Although it is possible to transform a spin Hamiltonian into a fermionic one, it is impractical for large-scale simulations since lattice symmetries are lost in the new Hamiltonian.}: $\text{H}\Phi$~\cite{Kawamura2017quant}, SPINPACK~\cite{Schule2017Spinpack}, and QuSpin~\cite{Weinbe2017QuspinAPytho}.

\begin{figure}
  % \includegraphics[width=0.9\linewidth]{./chapters/JOSS_6_64_3537_2021/02_operator_application.jpg}
  \centering
  \begin{tikzpicture}
      \begin{axis}[
          ybar,
          width=7.5cm, height=5.625cm,
          ylabel={\small Relative time},
          axis x line*=bottom,
          axis y line=left,
          bar width=0.25cm,
          xmin={[normalized]-0.5},
          legend entries={lattice-symmetries, QuSpin, SPINPACK},
          every axis legend/.append style={
              at={(0.5, -0.3)},
              anchor=north,
              line width=0.7pt,
              legend columns=-1,
              /tikz/every even column/.append style={column sep=0.3cm},
          },
          every axis/.append style={
              line width=1pt,
              tickwidth=0pt,
              tick style={line width=0.8pt},
              tick label style={font=\footnotesize},
              label style={font=\small},
              legend style={font=\small},
          },
          ymin={0.1},
          ytick={5,10,15,20},
          xtick=data,
          x tick label style={rotate=40, anchor=north east},
          every node near coord/.append style={rotate=90, anchor=west, font=\scriptsize},
          % nodes near coords={\pgfmathprintnumber[precision=2]{\pgfplotspointmeta}},
          symbolic x coords={
              $5 \times 5$,
              $5 \times 6$,
              $4 \times 8$,
              $5 \times 7$,
              $6 \times 6$,
              $5 \times 8$
          },
      ]
          \addplot [
              draw=black,
              fill=Pastel2-B,
              nodes near coords={\pgfmathprintnumber[fixed, fixed zerofill, precision=2]{\pgfplotspointmeta}}
          ] table [col sep=tab, meta=system, y expr=1.0, point meta=\thisrow{ls}]{chapters/JOSS_6_64_3537_2021/cn71/02_operator_application_1.dat};
          \addplot [
              draw=black,
              fill=Pastel2-C
          ] table [col sep=tab, y expr=\thisrow{quspin}/\thisrow{ls}]{chapters/JOSS_6_64_3537_2021/cn71/02_operator_application_1.dat};
          \addplot [
              draw=black,
              fill=Pastel2-D
          ] table [col sep=tab, y expr=\thisrow{spinpack}/\thisrow{ls}]{chapters/JOSS_6_64_3537_2021/spinpack/02_operator_application_1.dat};
      \end{axis}
  \end{tikzpicture}
  \caption{Performance of matrix-vector products in QuSpin, SPINPACK, and lattice-symmetries. For a Heisenberg Hamiltonian on square lattices of different sizes, we measure the time it takes to do a single matrix-vector product. Timings for lattice-symmetries are normalized to $1$ to show relative speedup compared to QuSpin, with absolute times in seconds listed as well. Depending on the system speedups over QuSpin vary between 5 and 22 times, but in all cases lattice-symmetries is significantly faster.}
  \label{fig:ls21:performance}
\end{figure}

$\text{H}\Phi$ implements a variety of Hamiltonians, works at both zero and finite temperatures, and supports multi-node computations. However, there are a few points in which lattice-symmetries improves upon $\text{H}\Phi$. Firstly, $\text{H}\Phi$ does not support arbitrary lattice symmetries. Secondly, it uses a custom input file format making it not user-friendly. Finally, since $\text{H}\Phi$ is an executable, it cannot be used as a library to develop new algorithms and applications.

SPINPACK is another popular solution for diagonalization of spin Hamiltonians. SPINPACK does support user-defined symmetries, unlike $\text{H}\Phi$, but its interface is even less user-friendly. Defining a lattice, Hamiltonian, and symmetries requires writing non-trivial amounts of C code. Finally, SPINPACK is slower than lattice-symmetries as illustrated in \autoref{fig:ls21:performance}.

QuSpin is much closer in functionality to lattice-symmetries. It is a high-level Python package, which natively supports (but is not limited to) spin systems, can employ user-defined lattice symmetries, and can also perform matrix-free calculations (where matrix elements are computed on the fly). However, QuSpin focuses mostly on ease of use and functionality rather than performance. In lattice-symmetries we follow UNIX philosophy~\cite{salus1994} and try to "do one thing but do it well". Even though lattice-symmetries uses essentially the same algorithms as QuSpin, careful implementation allows us to achieve an order of magnitude speedup, as shown in \autoref{fig:ls21:performance}. To achieve such performance, we make heavy use of Single Instruction Multiple Data (SIMD) instructions supported by modern processors. Vector Class Library~\cite{vectorclass} is used to write all performance-critical kernels, which currently limits the portability of lattice-symmetries to processors supporting x86-64 instruction sets.

lattice-symmetries is a library implemented in C++ and C. It provides two interfaces:
\begin{itemize}[topsep=0pt,noitemsep]
  \item Low-level C interface which can be used to implement ED and VMC applications with focus on performance.
  \item A higher-level Python wrapper which allows to easily test and prototype algorithms.
\end{itemize}

The library is easily installable via the Conda package manager.

The general workflow is as follows: the user starts by defining a few symmetry generators (\texttt{ls\_symmetry}/\texttt{Symmetry} in C/Python) from which \texttt{lattice-symmetries} automatically constructs the symmetry group (\texttt{ls\_group}/\texttt{Group} in C/Python). The user then proceeds to constructing the Hilbert space basis (\texttt{ls\_spin\_basis}/\texttt{SpinBasis} in C/Python). For some applications functionality provided by \texttt{SpinBasis} may be sufficient, but typically the user will construct one (or multiple) quantum mechanical operators (\texttt{ls\_operator}/\texttt{Operator} in C/Python) corresponding to the Hamiltonian and various observables. \texttt{lattice-symmetries} supports generic 1-, 2-, 3-, and 4-point operators. Examples of Hamiltonians that can be constructed include

\begin{equation*}
\begin{aligned}
    H &= \sum_{i, j} J_{ij} \boldsymbol\sigma_i \cdot \boldsymbol\sigma_j && \text{Heisenberg interaction,} \\
    H &= \sum_{i, j} J_{ij} \sigma^z_i \sigma^z_j + \sum_i h_i \sigma^x_i && \text{Ising model in transverse magnetic field,} \\
    H &= \sum_{i, j} \mathbf{D}_{ij} \left[ \boldsymbol\sigma_i \times \boldsymbol\sigma_j \right] && \text{Dzyaloshinskii-Moriya interaction.} \\
\end{aligned}
\end{equation*}

Here, $\boldsymbol\sigma$ denotes Pauli matrices, $J$ and $\mathbf{D}$ are various coupling constants, and sums over $i$ and $j$ can run over arbitrary used-defined geometries.

\texttt{Operator}s can be efficiently applied to vectors in the Hilbert space (i.e., wavefunctions). Also, in cases when the Hilbert space dimension is so big that the wavefunction cannot be written down explicitly (as a list of coefficients), \texttt{Operator} can be applied to individual spin configurations to implement Monte Carlo local estimators.

As an example of what can be done with lattice-symmetries, we implemented a standalone application for exact diagonalization studies of spin-1/2 systems, SpinED. By combining lattice-symmetries with the PRIMME eigensolver~\cite{Statho2010Primme}, it allows one to treat systems of at least 42 sites on a single node. SpinED is distributed as a statically-linked executable---one can download one file and immediately get started with physics. All in all, it makes large-scale ED more approachable for non-experts.

Notable research projects using lattice-symmetries and SpinED include~\cite{astrakhantsev2021,bagrov2020,westerhout2020}. Additionally, lattice-symmetries is being used to simulate quantum circuits~\cite{qsl2021}.

\section{Additional resources}

\QRListItem{https://github.com/twesterhout/lattice-symmetries/tree/v1}{The source code is available on GitHub. The repository contains scripts for running the benchmarks as well as raw timing data.}
